\chap{粮食与人口}
\section{引子}
生物都需要摄入能量维持自身运行。

\hfill ------自己说的 

\section{自然科学基本定律复习}
\theorem{热力学第一定律,能量守恒定律:}封闭系统内的能量总和保持不变,能量既不能凭空产生,也不能凭空消失,它只能从一种形式转化为另一种形式,或者从一个物体转移到另一个物体,在转移和转化的过程中,能量的总量不变;
\theorem 热力学第二定律,熵增定律:
\theorem 热力学第三定律,0K不能达到定律:
\theorem 热力学第零定律,热平衡定律:如果两个热力学系统均与第三个热力学系统处于热平衡,那么它们也必定处于热平衡 。也就是说热平衡是传递的。


\section{本书的理论基本假设}
\assume  社会规律遵循基本的物理学定律,尤其是热力学三定律。
\assume 生命的定义:
\assume 生物的定义:


\section{理论}
\theorem 生物都需要摄入能量维持自身运行。
\begin{proof}
    生物都是有基础生命运动的,基础生命运动都是需要消耗能量的,且若与外接有温差,温差会自发向外扩散。

    故而生命都需要摄入能量以维持自身生命活动。
\end{proof}
\newpage

% 下面这些内容都将使用定理或者引理描述
\section{粮食与人口}
\section{粮食与财富}
\section{人口与财富}
\section{信用及其衍生}
\section{粮食与暴力}
\section{人口与暴力}
\section{总结}

