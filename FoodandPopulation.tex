\chap{粮食与人口}
\section{引子}
生物都需要摄入能量维持自身运行。

\hfill ------自己说的 
\\
\\
\\

在编写本章的时候,我发现一个很核心的问题,就是目前关于生命或者生物的定义,在整个科学界还没有一个准确的公认的几乎没有疑义的定义,很多不同可科学家基于不同的角度尝试去定义过,但似乎都有一些小的争议,而我在这儿需要一个更准确的定义,但是本书的目的又不是长篇累牍的去讨论生命的定义本身,然后在生命的定义之上,尤其是比如以单细胞生物为例之上,会有组织的概念,这个概念所指的实体会由很多的单细胞组成,这个又和社会组织由很多的人组成很相似,这一个概念的外推很依赖于生命以及组织的准确定义,在我目前的看法中,基本上生命层次都是如此的两级的定义循环堆砌,由单细胞(生命)组成组织(生物组织),这个组织负责一定的功能,然后不同的组织,再组合成一个大的生命,如动物、植物,再由一个个的动物或植物个体组成小的组织或者社群,社群负责一部分功能,很多社群再组合成一个社会,这个社会就是一个生命体。而组织,包括生物组织和社会组织,其外在表现也很类似于单个的生命,类似于单细胞,类似于人类个体,类似于一个社会本身,只是他们所处的层级有所不同,但是都遵循相同的生存法则和本层级的另一些其他法则,如何有效的区分这些异同,也将是本书的重点,然后基于这些异同,描述一些外推的未来。

\section{自然科学基本定律复习}
\theorem{热力学第一定律,能量守恒定律:}封闭系统内的能量总和保持不变,能量既不能凭空产生,也不能凭空消失,它只能从一种形式转化为另一种形式,或者从一个物体转移到另一个物体,在转移和转化的过程中,能量的总量不变;
\theorem 热力学第二定律,熵增定律:
\theorem 热力学第三定律,0K不能达到定律:
\theorem 热力学第零定律,热平衡定律:如果两个热力学系统均与第三个热力学系统处于热平衡,那么它们也必定处于热平衡 。也就是说热平衡是传递的。


\section{本书的理论基本假设}
\assume  社会规律遵循基本的物理学定律,尤其是热力学三定律。
\assume 组织的定义:TODO:依赖于更低的元素的定义。
\assume 生命的定义:TODO:依赖于组织的定义
\assume 生物的定义:同生命的定义。


\section{理论}
\theorem 生物都需要摄入能量维持自身运行。
\begin{proof}
    生物都是有基础生命运动的,基础生命运动都是需要消耗能量的,且若与外接有温差,温差会自发向外扩散。

    故而生命都需要摄入能量以维持自身生命活动。
\end{proof}

\begin{theorem}
    生命的基本目标是生存和繁衍。
\end{theorem}
\begin{proof}
    TODO:
\end{proof}

\begin{theorem}
    生命是自私的。
\end{theorem}
\begin{proof}
    以为生命都需要摄入能量维持自身的生命活动运行,且生命的基本目标是生存和繁衍,故而生命的日常活动主要是为了维持自身存在及繁衍,即摄入外界的物质和能量为自己所用,只要生命哈爱活着,就必然其本能的去进行这一过程。故而说生命都市自私的。
\end{proof}

\begin{theorem}
    生命具有排他性。
\end{theorem}
\begin{proof}
    由生命的自私性,以及封闭系统内的能量总和不变,可以得知,生命具有排他性,一份能量,在同一个生态位上,为一个生命所摄入,则不能为另一个生命所摄入。
\end{proof}

\begin{theorem}
    生命的寿命有限。
\end{theorem}
\begin{proof}
    TODO:以无线时间线上小概率事件累计的可能性导致的必然发生来论证生命有寿命的基本理论。还必须用到组织的基本定义。
\end{proof}

\begin{theorem}
    在封闭系统内,总的能量能承载的生命数量是根据生命的具体组织形式体量来大致确定而有上限的。
\end{theorem}

这个理论其实很好理解与证明,只要想一下,在一个微生物培养皿内,微生物种群的数量在一定的时间内就会达到一个上限,然后逐步的死亡,后续会消耗完培养皿内的所有营养物质,然后所有种群的个体都全部死亡,培养皿上的微生物种群外观也是在经历微生物质数增长时的种群快速扩散到后面的种群外观长期不再产生变化。

在实际证明中发现并不是这么好证明,实际证明需要的参数有生命的寿命有限论,生命的是否繁殖的假设,以及生命的生存空间占据理论,其中,寿命论必须要依靠概率论的基本概念才能推导。
\begin{proof}
    情况一(培养皿模型):假设一个封闭系统中有总量为$m$的可摄入能量,其中有$1$中生物,可摄入这些能量,且摄入这些能量之后,会将这些能量以自身的生命活动所耗散,以及转换为不可为自身所重新摄入的能量,再假设该类生物单位时间的所需的耗散量为$n$,
\end{proof}

\newpage

% 下面这些内容都将使用定理或者引理描述
\section{粮食与人口}
\section{粮食与财富}
\section{人口与财富}
\section{信用及其衍生}
\section{粮食与暴力}
\section{人口与暴力}
\section{总结}

