\chap{介绍}
\section{思路来源}
我一直以来,都在思考,社会到底是怎么运行的,经济学当中的理性人的假设为什么总是反直觉的,甚至是与长期观测到的个体是不相符合的。社会运行的基本规律,经济运行的基本规律,组织架构的基本规律,是否可以像物理学那样不断的追根溯源,不断的追寻到一个或者几个最基本的假设假设上面,然后在这几项基本假设的基础上搭建出来一个模型,这个模型的自然演变的成百上千个可能模型当中,不仅包含了我们现在运行的社会模型,还能推导出由于某些细微初始条件不同所带来的即使是历史上,我们也从来没有经历过,记载过,甚至是想象过的模型。他们可能存在过,但是由于某些条件限制,最终走向了死亡。

说到了死亡,本书中要描写的一切生命体或者由生命体组成的组织,都将按照生命体那样去描述,而且,应该能够说明,组织与生命体的诸多相似之处,很多关于生命体的结论可以简单外推到组织系统当中。

继续说模型,目前我有一些浅显的思考,比如这个模型是否简单修改某些参数,就能套用《商子》那种法家的模型,以及后面儒家的模型,还有罗马的变迁,以及文艺复兴后的这种工业革命导致生产力加速发展的模型。

当然以上这些主要都是讨论一个组织内部的关系,也会有组织与组织群体与群体的关系,当然最重要的我们是要基于组织与环境的关系来描述这一切,也就是基于热力学第一定律:能量守恒定律来描述所有这一切,不知道是不是算是初次,但至少是我初次识图以物理学的基本定律来重新构建整个社会、经济的的基本规律,让很早就和物理分家,长期发展域局部的理论重新归为一统。


\section{目标读者}

\section{对读者的假设}



