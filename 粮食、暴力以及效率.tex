\documentclass[10pt,b5paper,twoside]{book}

%\usepackage{algorithm}  
%\usepackage{algorithmic}  
\usepackage{wallpaper} %使用wallpaper宏包

\usepackage[final]{pdfpages} %插入整页pdf
\usepackage[T1]{fontenc}
\usepackage{CJKutf8}
%\usepackage{CJK}
\usepackage[utf8]{inputenc}
\usepackage{amsmath}
\usepackage{amsfonts}
\usepackage{amssymb}
\usepackage{subfigure}
\usepackage{graphicx}
\usepackage{grffile}
\usepackage{epstopdf}%支持eps格式图
\usepackage[centerlast]{caption}
% \counterwithin{figure}{section} %图按照章节编号
\counterwithin{figure}{chapter} %图按照章节编号
% \counterwithin{equation}{section} %公式按照章节编号
\counterwithin{equation}{chapter} %公式按照章节编号
\counterwithin{table}{section} %表按照章节编号
%\usepackage{fourier}
\usepackage[left=2cm,right=2cm,top=2cm,bottom=2cm]{geometry}

\usepackage[export]{adjustbox}
\graphicspath{ {./images/} }

%设置缩进
\usepackage{indentfirst} 
\setlength{\parindent}{2em}

\usepackage{amsthm}
% 中文定理环境
\theoremstyle{plain}
% \indent 为了段前空两格
% \newtheorem{theorem}{\indent 定理}[section]
\newtheorem{assume}{\indent 假设}[chapter]
\newtheorem{theorem}{\indent 定理}[chapter]
\newtheorem{lemma}[theorem]{\indent 引理}
\newtheorem{proposition}[theorem]{\indent 命题}
\newtheorem{corollary}[theorem]{\indent 推论}
\newtheorem{definition}{\indent 定义}[section]
% \newtheorem{example}{\indent 例}[section]
% \newtheorem{example}{\indent 例}%[chapter]
\newtheorem{example}{\indent 例}%[chapter]
\newtheorem{remark}{\indent 注}[section]
\newenvironment{solution}{\begin{proof}[\indent\bf 解]}{\end{proof}}
% \renewcommand{\proofname}{\indent\bf 证明}


% % English theorem environment
% \newtheorem{theorem}{Theorem}[section]
% \newtheorem{lemma}[theorem]{Lemma}
% \newtheorem{proposition}[theorem]{Proposition}
% \newtheorem{corollary}[theorem]{Corollary}
% \newtheorem{definition}{Definition}[section]
% \newtheorem{remark}{Remark}[section]
% \newtheorem{example}{Example}[section]
% \newenvironment{solution}{\begin{proof}[Solution]}{\end{proof}}

% 附录
\usepackage[title]{appendix}

%设置行距
\linespread{1.4}

%设置章节显示
\usepackage[center]{titlesec}
%\titleformat{\chapter}
\definecolor{gray75}{gray}{0.75}
\newcommand{\hsp}{\hspace{20pt}}
\titleformat{\chapter}{\Huge\bfseries}{\begin{CJK}{UTF8}{gkai}第\,\thechapter\,章\hsp\textcolor{gray75}{|}\hsp\end{CJK}}{0pt}{\Huge\bfseries}

%% 重新定义章节字体,并同时更新例子编号
\newcommand{\chap}[1]{
	\begin{CJK}{UTF8}{gkai}\chapter{#1} \end{CJK} %更改字体
	\setcounter{example}{0} % 更改例子编号
}

\makeatletter %使\subsection中的内容左对齐
\renewcommand{\subsection}{\@startsection{subsection}{2}{0mm}
  {-\baselineskip}{0.5\baselineskip}{\bf\leftline}}
\renewcommand{\subsubsection}{\@startsection{subsubsection}{2}{0mm}
  {-\baselineskip}{0.5\baselineskip}{\bf\leftline}}

% 更改目录的章显示
\usepackage{titletoc}
% \titlecontents{chapter}[0pt]{\addvspace{1.5pt}\filright\bf}%
%                {\contentspush{第\,\thecontentslabel\,章\quad}}%
%                {}{\titlerule*[8pt]{.}\contentspage}
\titlecontents{chapter}[0pt]{\addvspace{1.5pt}\filright\bf}%
               {\contentspush{第\,\thecontentslabel\,章\quad}}%
               {}{\titlerule*[8pt]{.}\contentspage}
\setcounter{tocdepth}{2} %设置目录深度为只显示到section一级

% 更改页眉页脚为第几章形式
%\usepackage{fancyhdr}%导入fancyhdr包
%\pagestyle{fancy}
%\fancyhead{} % 初始化页眉
%\renewcommand{\chaptermark}[1]{\markboth{#1}{}}
\renewcommand{\chaptermark}[1]{\markboth{\begin{CJK}{UTF8}{gkai}第\,\thechapter\,章\, #1\end{CJK}}{}}
\renewcommand{\sectionmark}[1]{\markright{\thesection\, #1}}


%\fancyhead[LE]{\textsl{\rightmark}}
%\fancyhead[RO]{\textsl{\leftmark}}
%\renewcommand{\sectionmark}[1]{\markright{\thesection\,节\, #1}}
%\renewcommand{\chaptermark}[1]{\markright{第\,\thechapter\,章\, #1}{}}

%c++代码显示环境
\usepackage{listings}
\lstset{language=C++}

%算法伪代码环境
\usepackage[ruled,vlined]{algorithm2e}
%\usepackage[linesnumbered,lined,boxed,commentsnumbered]{algorithm2e}
%\usepackage{algorithm}
%\usepackage{algorithmic}

% 超链接设置及冲突解决
\usepackage{hyperref}
\hypersetup{unicode}
\hypersetup{colorlinks=true, linkcolor=blue, filecolor=magenta, urlcolor=cyan,}
\urlstyle{same}

% 使序和译序以及目录有书签
%\usepackage{lipsum}
%\usepackage{tocbibind}%为目录、参考文献和索引生成目录项
%\usepackage{bookmark}
%\hypersetup{CJKbookmarks=true}%让设置超链接后章节里面能写中文

%更改目录
\renewcommand{\contentsname}{目录}

% 更改算法显示为中文
\renewcommand{\algorithmcfname}{算法}

%更改图表的显示
\renewcommand{\figurename}{图}
\renewcommand{\tablename}{表}

\begin{document}
\begin{CJK}{UTF8}{gkai}

	\setcounter{page}{0}
	\pagenumbering{roman}
\begin{titlepage}
  \vspace*{1cm}
	\begin{center}
		{\Huge \textbf{粮食、暴力以及效率}\par}
		{\Huge \textbf{Food, Power and Efficiency}\par}
		%{\Huge\raggedright 图形瑰宝\par}
  		%\noindent\hrulefill\par
  		\vspace*{3cm}
  		{\Large\raggedleft 郭飞\par}
  		
  		\vfill
  		{\Large 2023年\par}
  		%{\Large\raggedleft Institute\par}
	\end{center}
\end{titlepage} 

\thispagestyle{empty}
% \mbox{}
% \newpage
% \thispagestyle{empty}
% 	\vspace*{5cm}
% 	\begin{center}
% 		\textit{谨以此书献给Heidi.} 
% 	\end{center}
\newpage
\thispagestyle{empty}
\clearpage
\mbox{}
\end{CJK}
%\begin{titlingpage}
%\maketitle
%\end{titlingpage}
%\includepdf{booktest (another copy).pdf}

%\newpage

%\maketitle

%\ThisCenterWallPaper{0.5}{./fig/cover.png}


\begin{CJK}{UTF8}{gbsn}
	
%\bookmark[page=1,level=0]{标题}
\newpage
\include{Preface}

\begin{CJK}{UTF8}{gkai}
\pdfbookmark{\contentsname}{Contents}
\tableofcontents
\end{CJK}
% \bookmark[dest=\HyperLocalCurrentHref, level=0]{目录}
%\pdfbookmark{目录}{contentsname}
%\bookmark[page=9,level=0]{目录}

\newpage
\setcounter{page}{0}
\pagenumbering{arabic}
\chap{介绍}
\section{思路来源}
我一直以来,都在思考,社会到底是怎么运行的,经济学当中的理性人的假设为什么总是反直觉的,甚至是与长期观测到的个体是不相符合的。社会运行的基本规律,经济运行的基本规律,组织架构的基本规律,是否可以像物理学那样不断的追根溯源,不断的追寻到一个或者几个最基本的假设假设上面,然后在这几项基本假设的基础上搭建出来一个模型,这个模型的自然演变的成百上千个可能模型当中,不仅包含了我们现在运行的社会模型,还能推导出由于某些细微初始条件不同所带来的即使是历史上,我们也从来没有经历过,记载过,甚至是想象过的模型。他们可能存在过,但是由于某些条件限制,最终走向了死亡。

说到了死亡,本书中要描写的一切生命体或者由生命体组成的组织,都将按照生命体那样去描述,而且,应该能够说明,组织与生命体的诸多相似之处,很多关于生命体的结论可以简单外推到组织系统当中。

继续说模型,目前我有一些浅显的思考,比如这个模型是否简单修改某些参数,就能套用《商子》那种法家的模型,以及后面儒家的模型,还有罗马的变迁,以及文艺复兴后的这种工业革命导致生产力加速发展的模型。

当然以上这些主要都是讨论一个组织内部的关系,也会有组织与组织群体与群体的关系,当然最重要的我们是要基于组织与环境的关系来描述这一切,也就是基于热力学第一定律:能量守恒定律来描述所有这一切,不知道是不是算是初次,但至少是我初次识图以物理学的基本定律来重新构建整个社会、经济的的基本规律,让很早就和物理分家,长期发展域局部的理论重新归为一统。


\section{目标读者}

\section{对读者的假设}




\chap{粮食与人口}
\section{引子}
生物都需要摄入能量维持自身运行。

\hfill ------自己说的 

\section{自然科学基本定律复习}
\theorem{热力学第一定律,能量守恒定律:}封闭系统内的能量总和保持不变,能量既不能凭空产生,也不能凭空消失,它只能从一种形式转化为另一种形式,或者从一个物体转移到另一个物体,在转移和转化的过程中,能量的总量不变;
\theorem 热力学第二定律,熵增定律:
\theorem 热力学第三定律,0K不能达到定律:
\theorem 热力学第零定律,热平衡定律:如果两个热力学系统均与第三个热力学系统处于热平衡,那么它们也必定处于热平衡 。也就是说热平衡是传递的。


\section{本书的理论基本假设}
\assume  社会规律遵循基本的物理学定律,尤其是热力学三定律。
\assume 生命的定义:
\assume 生物的定义:


\section{理论}
\theorem 生物都需要摄入能量维持自身运行。
\begin{proof}
    生物都是有基础生命运动的,基础生命运动都是需要消耗能量的,且若与外接有温差,温差会自发向外扩散。

    故而生命都需要摄入能量以维持自身生命活动。
\end{proof}
\newpage

% 下面这些内容都将使用定理或者引理描述
\section{粮食与人口}
\section{粮食与财富}
\section{人口与财富}
\section{信用及其衍生}
\section{粮食与暴力}
\section{人口与暴力}
\section{总结}


\chap{效率与组织}
\section{引子}
\section{效率与合作:组织的形成}
\section{组织的基本规律:存续与扩张}
\section{效率与科学}
\section{概率}
\section{总结}

\chap{暴力与分配}
\section{暴力对资源的争夺}
\section{暴力结果的存续:秩序的建立}
\section{暴力的转移:秩序的崩坏与重生}
\section{秩序与正义}
正义从来就是不存在的,正义只是建立秩序的暴力组织对于秩序的抽象化表达。
\section{总结}
\chap{已知的社会状态}
\section{引子}
\section{扩张型社会}
社会组织形态是为了更多的人口,更多的粮食,更强的暴力。

法家是这类社会构建的思想代表,局部的稳定的生产离水平下的不断扩张。
\subsection{兼并式扩张}
\subsection{再生产扩张}

\section{稳态型社会}
社会组织形态求问,不追求效率的提升,人口的粮食及暴力的扩张。

儒家是这类社会构建思想的典范。

单由于组织的基本定律,势必会崩坏。
\subsection{稳态型社会}
\subsection{震荡稳态型社会}

\section{效率提升型社会}

\section{总结}


\chap{一些可能已经死掉的社会状态}
\section{引子}
社会形态死掉的原因有很多,但是核心依旧是粮食与人口的在生产出现问题,暴力难以组织,效率以组织形态提升。
\section{缺乏组织的暴力}
\section{缺乏再生产}
\section{总结}
\chap{总结}

% \include{Number_Sets_and_Algebra}
% \include{Complex_Numbers}
% \include{Complex_Plane}
% \include{Quaternion_Algebra}
% \include{3D_Rotation_Transforms}
% \include{Quaternions_in_Space}
% \include{Conclusion}
% \include{Appendix_eigen}
% \include{References}
%\include{num_pro}
%\include{mod_trans}



\clearpage
\end{CJK}
\end{document}